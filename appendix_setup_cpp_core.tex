\section*{Περιβάλλοντα ανάπτυξης εφαρμογών σε C++}
Υπάρχουν πολλοί τρόποι με τους οποίους μπορεί κανείς να αναπτύξει εφαρμογές σε C++. Κατ' ελάχιστο θα πρέπει να έχει πρόσβαση σε έναν compiler της C++ και σε έναν επεξεργαστή κειμένου. Διαδεδομένοι compilers της C++ είναι ο GNU compiler collection που περιέχει τον compiler της C (gcc) και τον compiler της C++ (g++), o clang, o Microsoft Visual C++ compiler, o C++ compiler της Intel (icc), o C++ compiler της Apple και άλλοι. Ομοίως, υπάρχουν πολλοί επεξεργαστές κειμένου που διευκολύνουν τη συγγραφή κώδικα όπως ο Visual Studio Code της Microsoft, ο SublimeText, o Atom και άλλοι.

Εναλλακτικά, μπορούν να χρησιμοποιηθούν τα γνωστά ως ολοκληρωμένα συστήματα ανάπτυξης εφαρμογών (IDEs=Integrated Development Environments). Ένα IDE διευκολύνει τη δημιουργία έργων (projects), ενσωματώνει πολλά εργαλεία (εκσφαλμάτωσης, εκτίμησης απόδοσης κώδικα, ελέγχου κώδικα κ.α.) και επιτρέπει τη διαχείριση των πόρων της εφαρμογής μέσα από ένα γνώριμο περιβάλλον. Διαδεδομένα IDEs είναι τα:
\begin{itemize}
\item Microsoft Visual Studio
\item JetBrains CLion
\item Xcode (για ανάπτυξη εφαρμογών σε υπολογιστές της Apple)
\item Netbeans for C++
\item Eclipse CDT
\item Code::Blocks
\item CodeLite
\item Geany 
\item Dev-C++
\end{itemize}  

\subsection*{Visual Studio Code και g++}
Ένας εύχρηστος τρόπος ανάπτυξης εφαρμογών σε C++ είναι χρησιμοποιώντας τον επεξεργαστή κειμένου της Microsoft, Visual Studio Code και το μεταγλωττιστή g++. Και τα δύο λογισμικά είναι ελεύθερα διαθέσιμα και μπορούν να εγκατασταθούν και στα τρία πλέον διαδεδομένα λειτουργικά συστήματα (Windows, Linux και macOS).   

\subsubsection*{Εγκατάσταση σε Windows}
Η εγκατάσταση του g++ στα Windows μπορεί να γίνει από το MinGW Distro - nuwen.net \href{https://nuwen.net/mingw.html}{https://nuwen.net/mingw.html} ακολουθώντας τις οδηγίες στο \href{https://nuwen.net/mingw.html#install}{https://nuwen.net/mingw.html#install}. Η εγκατάσταση του Microsoft Visual Studio Code γίνεται πολύ απλά κατεβάζοντας και εκτελώντας το αντίστοιχο εκτελέσιμο από την ιστοσελίδα \href{https://code.visualstudio.com/}{https://code.visualstudio.com/}. Στη συνέχεια, μέσα από το Visual Studio Code πραγματοποιείται εγκατάσταση της επέκτασης (extention) C/C++, ms-vscode.cpptools.

%\subsubsection*{Εγκατάσταση σε Ubuntu Linux}
%
%\subsubsection*{Εγκατάσταση σε macOS}

\subsection*{Online C++ compilers}
Για το γρήγορο έλεγχο κώδικα μπορούν να χρησιμοποιηθούν ιστοσελίδες που διαθέτουν υπηρεσίες συγγραφής κώδικα, απομακρυσμένης μεταγλώττισης και εκτέλεσης του κώδικα. Ορισμένες σχετικές ιστοσελίδες είναι οι ακόλουθες:
\begin{itemize}
\item C++ shell (http://cpp.sh/)
\item Coliru (https://coliru.stacked-crooked.com/)
\item OnlineGDB (https://www.onlinegdb.com/)
\end{itemize}  


