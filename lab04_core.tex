\section{Εισαγωγή}
Οι γραμμικές λίστες είναι δομές δεδομένων που επιτρέπουν την αποθήκευση και την προσπέλαση στοιχείων έτσι ώστε τα στοιχεία να βρίσκονται σε μια σειρά με σαφώς ορισμένη την έννοια της θέσης καθώς και το ποιο στοιχείο προηγείται και ποιο έπεται καθενός. Σε χαμηλού επιπέδου γλώσσες προγραμματισμού όπως η C η υλοποίηση γραμμικών λιστών είναι ευθύνη του προγραμματιστή. Από την άλλη μεριά, γλώσσες υψηλού επιπέδου όπως η C++, η Java, η Python κ.α. προσφέρουν έτοιμες υλοποιήσεις γραμμικών λιστών. Ωστόσο, η γνώση υλοποίησης των συγκεκριμένων δομών (όπως και άλλων) αποτελεί βασική ικανότητα η οποία αποκτά ιδιαίτερη χρησιμότητα όταν ζητούνται εξειδικευμένες υλοποιήσεις. Για το λόγο αυτό στο συγκεκριμένο εργαστήριο θα παρουσιαστούν οι υλοποιήσεις γραμμικών λιστών αλλά και οι ενσωματωμένες δυνατότητες της C++ μέσω της STL.

\section{Γραμμικές λίστες}
Υπάρχουν δύο βασικοί τρόποι αναπαράστασης γραμμικών λιστών, η στατική αναπαράσταση η οποία γίνεται με τη χρήση πινάκων και η αναπαράσταση με συνδεδεμένη λίστα η οποία γίνεται με τη χρήση δεικτών. 

\subsection{Στατικές γραμμικές λίστες}
Στη στατική γραμμική λίστα τα δεδομένα αποθηκεύονται σε ένα πίνακα. Κάθε στοιχείο της στατικής λίστας μπορεί να προσπελαστεί με βάση τη θέση του στον ίδιο σταθερό χρόνο με όλα τα άλλα στοιχεία άσχετα με τη θέση στην οποία βρίσκεται (τυχαία προσπέλαση). Ο κώδικας υλοποίησης μιας στατικής λίστας με μέγιστη χωρητικότητα 50.000 στοιχείων παρουσιάζεται στη συνέχεια.

\lstinputlisting[caption = Υλοποίηση στατικής γραμμικής λίστας (static\_list.cpp)]{lab04/static_list.cpp}

\lstinputlisting[caption = Παράδειγμα με στατική γραμμική λίστα (list1.cpp)]{lab04/list1.cpp}

\lstinputlisting[style=DOS]{lab04/list1.out}


\subsection{Συνδεδεμένες γραμμικές λίστες}
\lstinputlisting[caption = Υλοποίηση συνδεδεμένης γραμμικής λίστας (linked\_list.cpp)]{lab04/linked_list.cpp}

\lstinputlisting[caption = Παράδειγμα με συνδεδεμένη γραμμική λίστα (list2.cpp)]{lab04/list2.cpp}

\lstinputlisting[style=DOS]{lab04/list2.out}

\subsection{Γραμμικές λίστες της STL}
\subsubsection{list}
\subsubsection{forwardlist}
\subsubsection{vector}


\section{Παραδείγματα}
\subsection{Παράδειγμα 1}
Γράψτε ένα πρόγραμμα που να ελέγχεται από το ακόλουθο μενού και να πραγματοποιεί τις λειτουργίες που περιγράφονται σε μια απλά συνδεδεμένη λίστα με ακεραίους
\begin{enumerate}
\item 1. Show items
\item 2. Insert item at given position 
\item 3. Delete item at given position
\item 4. Delete item
\item 5. Exit
\end{enumerate}

\subsection{Παράδειγμα 2}
Έστω μια υποθετική τράπεζα. Για κάθε πελάτη έστω ότι η τράπεζα διατηρεί σε ένα αρχείο το ονοματεπώνυμο του και το υπόλοιπο του λογαριασμού του. Για τις ανάγκες της άσκησης θα πρέπει να δημιουργηθούν τυχαίοι πελάτες ως εξής: το όνομα κάθε πελάτη να αποτελείται από 10 γράμματα που θα επιλέγονται με τυχαίο τρόπο από τα γράμματα της αγγλικής αλφαβήτου και το δε υπόλοιπο κάθε πελάτη να είναι ένας τυχαίος αριθμός από το 0 μέχρι το 5.000. 
Θα παρουσιαστούν τέσσερις εκδόσεις του ίδιου προγράμματος. Η μεν πρώτη θα υλοποιείται με στατική λίστα,  η δεύτερη με συνδεδεμένη λίστα η τρίτη με τη στατική γραμμική λίστα της C++, std::vector και η τέταρτη με τη συνδεδεμένη λίστα της C++, std::list. Και στις τέσσερις περιπτώσεις το πρόγραμμα θα πραγματοποιεί τις ακόλουθες λειτουργίες:

\begin{itemize}[noitemsep]
\item Θα δημιουργεί μια λίστα με 40.000 τυχαίους πελάτες.
\item Θα υπολογίζει το άθροισμα των υπολοίπων από όλους τους πελάτες που το όνομά τους ξεκινά με το χαρακτήρα Α.
\item Θα προσθέτει για κάθε πελάτη που το όνομά του ξεκινά με το χαρακτήρα Α στην αμέσως επόμενη θέση έναν πελάτη με όνομα το αντίστροφο όνομα του πελάτη και το ίδιο υπόλοιπο λογαριασμού.
\item Θα διαγράφει όλους τους πελάτες που το όνομά τους ξεκινά με το χαρακτήρα Β.
\end{itemize}

\lstinputlisting[caption = Παράδειγμα με συνδεδεμένη γραμμική λίστα (lab04\_ex1.cpp)]{lab04/lab04_ex1.cpp}

\lstinputlisting[style=DOS]{lab04/lab04_ex1.out}

\section{Ασκήσεις}
\begin{enumerate}
\item Έστω η συνδεδεμένη λίστα που παρουσιάστηκε στον κώδικα ΧΧ. Προσθέστε μια συνάρτηση που μια λίστα ακεραίων στην οποία τα στοιχεία της είναι ταξινομημένα από το μικρότερο στο μεγαλύτερο να προσθέτει ένα ακόμα στοιχείο στην κατάλληλη θέση έτσι ώστε η λίστα να παραμένει ταξινομημένη.
\item Έστω η συνδεδεμένη λίστα που παρουσιάστηκε στον κώδικα ΧΧ. Προσθέστε μια συνάρτηση που να αντιστρέφει τη λίστα.
\item Υλοποιήστε τους κώδικες της στατικής ΧΧ και της συνδεδεμένης λίστας ΧΧ με κλάσεις. Τροποποιήστε το παράδειγμα 1 έτσι ώστε να δίνεται επιλογή στο χρήστη να χρησιμοποιήσει είτε τη στατική είτε τη συνδεδεμένη λίστα προκειμένου να εκτελέσει τις ίδιες λειτουργίες πάνω σε μια λίστα. 
\item Υλοποιήστε μια κυκλικά συνδεδεμένη λίστα. Η κυκλική λίστα είναι μια απλά συνδεδεμένη λίστα στην οποία το τελευταίο στοιχείο της λίστας δείχνει στο πρώτο στοιχείο της λίστας. Η υλοποίηση θα πρέπει να συμπεριλαμβάνει και δύο δείκτες, έναν που να δείχνει στο πρώτο στοιχείο της λίστας και έναν που να δείχνει στο τελευταίο στοιχείο της λίστας. Προσθέστε τις απαιτούμενες λειτουργίες έτσι ώστε η λίστα να παρέχονται οι ακόλουθες λειτουργίες: εμφάνιση λίστας, εισαγωγή στοιχείου, διαγραφή στοιχείου, εμφάνιση πλήθους στοιχείων, εύρεση στοιχείου. Γράψτε πρόγραμμα που να δοκιμάζει τις λειτουργίες της λίστας.
\end{enumerate}


\begin{thebibliography}{9}

\bibitem{hackernoon}
Stable Sorting, \href{https://hackernoon.com/stable-sorting-677453884792}{https://hackernoon.com/stable-sorting-677453884792}

\end{thebibliography}

