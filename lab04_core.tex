\section{Εισαγωγή}
Οι γραμμικές λίστες είναι δομές δεδομένων που επιτρέπουν την αποθήκευση και την προσπέλαση στοιχείων έτσι ώστε τα στοιχεία να βρίσκονται σε μια σειρά με σαφώς ορισμένη την έννοια της θέσης καθώς και το ποιο στοιχείο προηγείται και ποιο έπεται καθενός. Σε χαμηλού επιπέδου γλώσσες προγραμματισμού όπως η C η υλοποίηση γραμμικών λιστών είναι ευθύνη του προγραμματιστή. Από την άλλη μεριά, γλώσσες υψηλού επιπέδου όπως η C++, η Java, η Python κ.α. προσφέρουν έτοιμες υλοποιήσεις γραμμικών λιστών. Ωστόσο, η γνώση υλοποίησης των συγκεκριμένων δομών (όπως και άλλων) αποτελεί βασική ικανότητα η οποία αποκτά ιδιαίτερη χρησιμότητα όταν ζητούνται εξειδικευμένες υλοποιήσεις. Στο συγκεκριμένο εργαστήριο θα παρουσιαστούν δύο πιθανές υλοποιήσεις γραμμικών λιστών (στατικής λίστας και απλά συνδεδεμένης λίστας) καθώς και οι ενσωματωμένες δυνατότητες της C++ μέσω containers της STL όπως το vector, το list και άλλα.

\section{Γραμμικές λίστες}
Υπάρχουν δύο βασικοί τρόποι αναπαράστασης γραμμικών λιστών, η στατική αναπαράσταση η οποία γίνεται με τη χρήση πινάκων και η αναπαράσταση με συνδεδεμένη λίστα η οποία γίνεται με τη χρήση δεικτών. 

\subsection{Στατικές γραμμικές λίστες}
Στη στατική γραμμική λίστα τα δεδομένα αποθηκεύονται σε ένα πίνακα. Κάθε στοιχείο της στατικής λίστας μπορεί να προσπελαστεί με βάση τη θέση του στον ίδιο σταθερό χρόνο με όλα τα άλλα στοιχεία άσχετα με τη θέση στην οποία βρίσκεται (τυχαία προσπέλαση). Ο κώδικας υλοποίησης μιας στατικής λίστας με μέγιστη χωρητικότητα 50.000 στοιχείων παρουσιάζεται στη συνέχεια.

\lstinputlisting[caption = Υλοποίηση στατικής γραμμικής λίστας (static\_list.cpp),label=lst:static_list.cpp, multicols=2]{lab04/static_list.cpp}

\lstinputlisting[caption = Παράδειγμα με στατική γραμμική λίστα (list1.cpp),label=lst:list1.cpp, multicols=2]{lab04/list1.cpp}

\lstinputlisting[style=DOS]{lab04/list1.out}

\paragraph{Εξαιρέσεις στη C++} Στους κώδικες που προηγήθηκαν καθώς και σε επόμενους γίνεται χρήση εξαιρέσεων (exceptions) για να σηματοδοτηθούν γεγονότα τα οποία αφορούν έκτακτες καταστάσεις που το πρόγραμμα θα πρέπει να διαχειρίζεται. Για παράδειγμα, όταν επιχειρηθεί η προσπέλαση ενός στοιχείου σε μια θέση εκτός των ορίων της λίστας (π.χ. ενέργεια 5 στον κώδικα \ref{lst:list1.cpp}) τότε γίνεται throw ένα exception out\_of\_range το οποίο θα πρέπει να συλληφθεί (να γίνει catch) από τον κώδικα που καλεί τη συνάρτηση που προκάλεσε το throw exception. Περισσότερες πληροφορίες για τα exceptions και τον χειρισμό τους μπορούν να αναζητηθούν στην αναφορά \cite{cppexceptions}.\\

\noindent Σχετικά με τις στατικές γραμμικές λίστες ισχύει ότι έχουν τα ακόλουθα πλεονεκτήματα:
\begin{itemize}[nolistsep]
\item Εύκολη υλοποίηση. 
\item Σταθερός χρόνος, $O(1)$, εντοπισμού στοιχείου με βάση τη θέση του.
\item Γραμμικός χρόνος, $O(n)$, για αναζήτηση ενός στοιχείου ή λογαριθμικός χρόνος, $O(log(n))$, αν τα στοιχεία είναι ταξινομημένα.
\end{itemize}

\noindent Ωστόσο, οι στατικές γραμμικές λίστες έχουν και μειονεκτήματα τα οποία παρατίθενται στη συνέχεια:
\begin{itemize}[nolistsep]
\item Δέσμευση μεγάλου τμήματος μνήμης ακόμη και όταν η λίστα είναι άδεια ή περιέχει λίγα στοιχεία. 
\item Επιβολή άνω ορίου στα δεδομένα τα οποία μπορεί να δεχθεί (ο περιορισμός αυτός μπορεί να ξεπεραστεί με συνθετότερη υλοποίηση που αυξομειώνει το μέγεθος του πίνακα υποδοχής όταν αυτό απαιτείται).
\item Γραμμικός χρόνος $O(n)$ για εισαγωγή και διαγραφή στοιχείων του πίνακα.
\end{itemize}


\subsection{Συνδεδεμένες γραμμικές λίστες}
Η συνδεδεμένη γραμμική λίστα αποτελείται από μηδέν ή περισσότερους κόμβους. Κάθε κόμβος περιέχει δεδομένα και έναν ή περισσότερους δείκτες σε άλλους κόμβους της συνδεδεμένης λίστας. Συχνά χρησιμοποιείται ένας πρόσθετος κόμβος με όνομα head (κόμβος κεφαλής) που δείχνει στο πρώτο στοιχείο της λίστας και μπορεί να περιέχει επιπλέον πληροφορίες όπως το μήκος της. Στη συνέχεια παρουσιάζεται ο κώδικας που υλοποιεί μια απλά συνδεδεμένη λίστα.

\lstinputlisting[caption = Υλοποίηση συνδεδεμένης γραμμικής λίστας (linked\_list.cpp),label=lst:linked_list.cpp, multicols=2]{lab04/linked_list.cpp}

\lstinputlisting[caption = Παράδειγμα με συνδεδεμένη γραμμική λίστα (list2.cpp), multicols=2]{lab04/list2.cpp}

\lstinputlisting[style=DOS]{lab04/list2.out}


\noindent Οι συνδεδεμένες γραμμικές λίστες έχουν τα ακόλουθα πλεονεκτήματα:
\begin{itemize}[nolistsep]
\item Καλή χρήση του αποθηκευτικού χώρου (αν και απαιτείται περισσότερος χώρος για την αποθήκευση κάθε κόμβου λόγω των δεικτών). 
\item Σταθερός χρόνος, $O(1)$, για την εισαγωγή και διαγραφή στοιχείων.
\end{itemize}

\noindent Από την άλλη μεριά τα μειονεκτήματα των συνδεδεμένων λιστών είναι τα ακόλουθα:
\begin{itemize}[nolistsep]
\item Συνθετότερη υλοποίηση.
\item Δεν επιτρέπουν την απευθείας μετάβαση σε κάποιο στοιχείο με βάση τη θέση του.
\end{itemize}

Οι αναφορές \cite{csstanford103} και \cite{csstanford105} παρέχουν χρήσιμες πληροφορίες και ασκήσεις σχετικά με τις συνδεδεμένες λίστες και το ρόλο των δεικτών στην υλοποίησή τους.

\subsection{Γραμμικές λίστες της STL}
Τα containers της STL που μπορούν να λειτουργήσουν ως διατεταγμένες συλλογές (ordered collections) είναι τα ακόλουθα: vector, deque, arrays, list, forward\_list και bitset. 

\subsubsection{Vectors}
Τα vectors αλλάζουν αυτόματα μέγεθος καθώς προστίθενται ή αφαιρούνται στοιχεία σε αυτά. Τα δεδομένα τους τοποθετούνται σε συνεχόμενες θέσεις μνήμης. Περισσότερες πληροφορίες για τα vectors μπορούν να βρεθούν στις αναφορές \cite{g4gvector} και \cite{codecogsvector}. Στο ακόλουθο παράδειγμα παρουσιάζονται 4 διαφορετικοί τρόποι με τους οποίους μπορεί να προσπελαστεί το πρώτο και το τελευταίο στοιχείο του διανύσματος καθώς και η δυνατότητα ελέγχου με τον τελεστή της ισότητας σχετικά με το αν δύο διανύσματα είναι ίσα.

\lstinputlisting[caption = Παράδειγμα με vectors (vector.cpp), multicols=2]{lab04/vector.cpp}

\lstinputlisting[style=DOS]{lab04/vector.out}


\subsubsection{Deques}
Τα deques (double ended queues = ουρές με δύο άκρα) είναι παρόμοια με τα vectors αλλά μπορούν να προστεθούν ή να διαγραφούν στοιχεία τόσο από την αρχή όσο και από το τέλος τους. Περισσότερες πληροφορίες για τα deques μπορούν να βρεθούν στην αναφορά \cite{g4gdeque}. Στο παράδειγμα που ακολουθεί εισάγονται σε ένα deque εναλλάξ στο αριστερό και στο δεξί άκρο οι άρτιοι και οι περιττοί ακέραιοι αριθμοί στο διάστημα [1,20].

\lstinputlisting[caption = Παράδειγμα με deque (deque.cpp), multicols=2]{lab04/deque.cpp}

\lstinputlisting[style=DOS]{lab04/deque.out}


\subsubsection{Arrays}
Τα arrays εισήχθησαν στη C++11 με στόχο να αντικαταστήσουν τους απλούς πίνακες της C. Κατά τη δήλωση ενός array προσδιορίζεται και το μέγεθός του. Περισσότερες πληροφορίες για τα arrays μπορούν να βρεθούν στην αναφορά \cite{g4garray}. Στο ακόλουθο παράδειγμα δημιουργείται ένα array με 5 πραγματικές τιμές, ταξινομείται και εμφανίζεται.

\lstinputlisting[caption = Παράδειγμα με array (array.cpp), multicols=2]{lab04/array.cpp}

\lstinputlisting[style=DOS]{lab04/array.out}


\subsubsection{Lists}
Οι lists είναι διπλά συνδεδεμένες λίστες. Δηλαδή κάθε κόμβος της λίστας διαθέτει έναν δείκτη προς το επόμενο και έναν δείκτη προς το προηγούμενο στοιχείο στη λίστα. Περισσότερες πληροφορίες για τις lists μπορούν να βρεθούν στην αναφορά \cite{g4glist}. Στο παράδειγμα που ακολουθεί μια διπλά συνδεδεμένη λίστα διανύεται από δεξιά προς τα αριστερά και από αριστερά προς τα δεξιά στην ίδια επανάληψη.

\lstinputlisting[caption = Παράδειγμα με list (forward\_list.cpp), multicols=2]{lab04/list.cpp}

\lstinputlisting[style=DOS]{lab04/list.out}


\subsubsection{Forward Lists}
Οι forward lists (λίστες προς τα εμπρός) είναι απλά συνδεδεμένες λίστες με κάθε κόμβο να διαθέτει έναν δείκτη προς το επόμενο στοιχείο της λίστας. Περισσότερες πληροφορίες για τις forward lists μπορούν να βρεθούν στις αναφορές \cite{g4gforwardlist1} και \cite{g4gforwardlist2}. Ακολουθεί ένα παράδειγμα που αντιστρέφει μια απλά συνδεδεμένη λίστα στην οποία έχουν πριν προστεθεί στοιχεία.

\lstinputlisting[caption = Παράδειγμα με forward\_list (forward\_list.cpp), multicols=2]{lab04/forward_list.cpp}

\lstinputlisting[style=DOS]{lab04/forward_list.out}


\subsubsection{Bitset}
Τα bitsets είναι πίνακες με λογικές τιμές τις οποίες αποθηκεύουν με αποδοτικό τρόπο καθώς για κάθε λογική τιμή απαιτείται μόνο 1 bit. Το μέγεθος ενός bitset πρέπει να είναι γνωστό κατά τη μεταγλώττιση. Μια ιδιαιτερότητά του είναι ότι οι δείκτες θέσης που χρησιμοποιούνται για την αναφορά στα στοιχεία του ξεκινούν την αρίθμησή τους με το μηδέν από δεξιά και αυξάνονται προς τα αριστερά. Για παράδειγμα ένα bitset με τιμές 101011 έχει την τιμή 1 στις θέσεις 0,1,3,5 και 0 στις θέσεις 2 και 4. Περισσότερες πληροφορίες για τα bitsets μπορούν να βρεθούν στις αναφορές \cite{g4gbitset1} και \cite{g4gbitset2}. Ακολουθεί ένα παράδειγμα που εμφανίζει χρησιμοποιώντας 5 δυαδικά ψηφία τους ακέραιους αριθμούς από το 0 μέχρι το 7.

\lstinputlisting[caption = Παράδειγμα με bitset (bitset.cpp)]{lab04/bitset.cpp}

\lstinputlisting[style=DOS]{lab04/bitset.out}

\section{Παραδείγματα}
\subsection{Παράδειγμα 1}
Γράψτε ένα πρόγραμμα που να ελέγχεται από το ακόλουθο μενού και να πραγματοποιεί τις λειτουργίες που περιγράφονται σε μια απλά συνδεδεμένη λίστα με ακεραίους.
\begin{enumerate}[nolistsep]
\item Εμφάνιση στοιχείων λίστας. (Show list)
\item Εισαγωγή στοιχείου στο πίσω άκρο της λίστας. (Insert item (back))
\item Εισαγωγή στοιχείου σε συγκεκριμένη θέση. (Insert item (at position)) 
\item Διαγραφή στοιχείου σε συγκεκριμένη θέση. (Delete item (from position))
\item Διαγραφή όλων των στοιχείων που έχουν την τιμή. (Delete all items having value)
\item Έξοδος. (Exit)
\end{enumerate}

\lstinputlisting[caption = Έλεγχος συνδεδεμένης λίστας ακεραίων μέσω μενού (lab04\_ex1.cpp), multicols=2]{lab04/lab04_ex1.cpp}

\lstinputlisting[style=DOS]{lab04/lab04_ex1.out}


\subsection{Παράδειγμα 2}
Έστω μια τράπεζα που διατηρεί για κάθε πελάτη της τον κωδικό του και το υπόλοιπο του λογαριασμού του. Για τις ανάγκες του παραδείγματος θα δημιουργηθούν τυχαίοι πελάτες ως εξής: ο κωδικός κάθε πελάτη θα αποτελείται από 10 σύμβολα που θα επιλέγονται με τυχαίο τρόπο από τα γράμματα της αγγλικής αλφαβήτου και το υπόλοιπο κάθε πελάτη θα είναι ένας τυχαίος ακέραιος αριθμός από το 0 μέχρι το 5.000. Το πρόγραμμα θα πραγματοποιεί τις ακόλουθες λειτουργίες: 
\begin{enumerate}[noitemsep,label=\Alph*]
\item Θα δημιουργεί μια λίστα με 40.000 τυχαίους πελάτες.
\item Θα υπολογίζει το άθροισμα των υπολοίπων από όλους τους πελάτες που ο κωδικός τους ξεκινά με το χαρακτήρα Α.
\item Θα προσθέτει για κάθε πελάτη που ο κωδικός του ξεκινά με το χαρακτήρα G στην αμέσως επόμενη θέση έναν πελάτη με κωδικό το αντίστροφο κωδικό του πελάτη και το ίδιο υπόλοιπο λογαριασμού.
\item Θα διαγράφει όλους τους πελάτες που ο κωδικός τους ξεκινά με το χαρακτήρα Β.
\end{enumerate}
Τα δεδομένα θα αποθηκεύονται σε μια συνδεδεμένη λίστα πραγματοποιώντας χρήση του κώδικα \ref{lst:linked_list.cpp} καθώς και άλλων συναρτήσεων που επιτρέπουν την αποδοτικότερη υλοποίηση των παραπάνω ερωτημάτων.

\lstinputlisting[caption = Λίστα πελατών για το ίδιο πρόβλημα (lab04\_ex2.cpp), multicols=2]{lab04/lab04_ex2.cpp}

\lstinputlisting[style=DOS]{lab04/lab04_ex2.out}

Αν στη θέση της συνδεδεμένης λίστας του κώδικα \ref{lst:linked_list.cpp} χρησιμοποιηθεί η στατική λίστα του κώδικα \ref{lst:static_list.cpp} ή ένα vector ή ένα list της STL τα αποτελέσματα θα είναι τα ίδια αλλά η απόδοση στα επιμέρους ερωτήματα του παραδείγματος θα διαφέρει όπως φαίνεται στον πίνακα \ref{tbl:lists}. Ο κώδικας που παράγει τα αποτελέσματα βρίσκεται στο αρχείο lab04/lab04\_ex2\_x4.cpp.

\begin{table}[!htb]
\centering
\begin{tabular}{|l|c|c|c|c|}
\hline
            & Ερώτημα A     & Ερώτημα B     & Ερώτημα Γ     & Ερώτημα Δ     \\ \hline
Συνδεδεμένη  λίστα & 0.030 & 0.001 & 0.002 & 0.001 \\ \hline
Στατική λίστα  & 0.034 & 0.003 & 0.642 & 0.671 \\ \hline
std::vector     & 0.033 & 0.002 & 0.543 & 0.519 \\ \hline
std::list      & 0.033 & 0.002 & 0.002 & 0.001 \\ \hline
\end{tabular}
\caption{Χρόνοι εκτέλεσης σε δευτερόλεπτα των ερωτημάτων του παραδείγματος 2 ανάλογα με τον τρόπο υλοποίησης της λίστας}
\label{tbl:lists}
\end{table}



\section{Ασκήσεις}
\begin{enumerate}[nolistsep]
\item Έστω η συνδεδεμένη λίστα που παρουσιάστηκε στον κώδικα \ref{lst:linked_list.cpp}. Προσθέστε μια συνάρτηση έτσι ώστε για  μια λίστα ταξινομημένων στοιχείων από το μικρότερο προς το μεγαλύτερο, να προσθέτει ένα ακόμα στοιχείο στην κατάλληλη θέση έτσι ώστε η λίστα να παραμένει ταξινομημένη.
\item Έστω η συνδεδεμένη λίστα που παρουσιάστηκε στον κώδικα \ref{lst:linked_list.cpp}. Προσθέστε μια συνάρτηση που να αντιστρέφει τη λίστα.
\item Υλοποιήστε τη στατική λίστα (κώδικας \ref{lst:static_list.cpp}) και τη συνδεδεμένη λίστα (κώδικας \ref{lst:linked_list.cpp}) με κλάσεις. Τροποποιήστε το παράδειγμα 1 έτσι ώστε να δίνεται επιλογή στο χρήστη να χρησιμοποιήσει είτε τη στατική είτε τη συνδεδεμένη λίστα προκειμένου να εκτελέσει τις ίδιες λειτουργίες πάνω σε μια λίστα. 
\item Υλοποιήστε μια κυκλικά συνδεδεμένη λίστα. Η κυκλική λίστα είναι μια απλά συνδεδεμένη λίστα στην οποία το τελευταίο στοιχείο της λίστας δείχνει στο πρώτο στοιχείο της λίστας. Η υλοποίηση θα πρέπει να συμπεριλαμβάνει και δύο δείκτες, έναν που να δείχνει στο πρώτο στοιχείο της λίστας και έναν που να δείχνει στο τελευταίο στοιχείο της λίστας. Προσθέστε τις απαιτούμενες λειτουργίες έτσι ώστε η λίστα να παρέχει τις ακόλουθες λειτουργίες: εμφάνιση λίστας, εισαγωγή στοιχείου, διαγραφή στοιχείου, εμφάνιση πλήθους στοιχείων, εύρεση στοιχείου. Γράψτε πρόγραμμα που να δοκιμάζει τις λειτουργίες της λίστας.
\end{enumerate}

\begin{thebibliography}{9}
\bibitem{cppexceptions}
C++ Tutorial-exceptions-2017 by K. Hong, \href{http://www.bogotobogo.com/cplusplus/exceptions.php}{http://www.bogotobogo.com/cplusplus/exceptions.php}.
\bibitem{csstanford103}
Linked List Basics by N. Parlante, \href{http://cslibrary.stanford.edu/103/}{http://cslibrary.stanford.edu/103/}.
\bibitem{csstanford105}
Linked List Problems by N. Parlante, \href{http://cslibrary.stanford.edu/105/}{http://cslibrary.stanford.edu/105/}.
\bibitem{g4gvector}
Geeks for Geeks, Vector in C++ STL, \href{http://www.geeksforgeeks.org/vector-in-cpp-stl/}{http://www.geeksforgeeks.org/vector-in-cpp-stl/}.
\bibitem{codecogsvector}
Codecogs, Vector, a random access dynamic container, \href{http://www.codecogs.com/library/computing/stl/containers/sequences/vector.php}{http://www.codecogs.com/library/computing}.
\bibitem{g4gdeque}
Geeks for Geeks, Deque in C++ STL, \href{http://www.geeksforgeeks.org/deque-cpp-stl/}{http://www.geeksforgeeks.org/deque-cpp-stl/}.
\bibitem{g4garray}
Geeks for Geeks, Array class in C++ STL \href{http://www.geeksforgeeks.org/array-class-c/}{http://www.geeksforgeeks.org/array-class-c/}.
\bibitem{g4glist}
Geeks for Geeks, List in C++ STL  \href{http://www.geeksforgeeks.org/list-cpp-stl/}{http://www.geeksforgeeks.org/list-cpp-stl/}
\bibitem{g4gforwardlist1}
Geeks for Geeks, Forward List in C++ (Set 1) \href{http://www.geeksforgeeks.org/forward-list-c-set-1-introduction-important-functions/}{http://www.geeksforgeeks.org/forward-list-c-set-1-introduction-important-functions/}
\bibitem{g4gforwardlist2}
Geeks for Geeks, Forward List in C++ (Set 2) \href{http://www.geeksforgeeks.org/forward-list-c-set-2-manipulating-functions/}{http://www.geeksforgeeks.org/forward-list-c-set-2-manipulating-functions/}
\bibitem{g4gbitset1}
Geeks for Geeks, C++ bitset and its application, \href{http://www.geeksforgeeks.org/c-bitset-and-its-application/}{http://www.geeksforgeeks.org/c-bitset-and-its-application/}
\bibitem{g4gbitset2}
Geeks for Geeks, C++ bitset interesting facts, \href{http://www.geeksforgeeks.org/c-bitset-interesting-facts/}{http://www.geeksforgeeks.org/c-bitset-interesting-facts/}
\end{thebibliography}

