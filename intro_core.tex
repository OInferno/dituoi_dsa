Το παρόν σύγγραμμα χωρίζεται σε εννέα κεφάλαια (εργαστήρια) που υποστηρίζουν το εργαστηριακό τμήμα του μαθήματος ``Δομές δεδομένων και αλγόριθμοι''. Η γλώσσα προγραμματισμού που χρησιμοποιείται είναι η C++ συμπεριλαμβανομένης της βιβλιοθήκης STL (Standard Template Library) ενώ παράλληλα γίνεται χρήση νέων δυνατοτήτων που έχουν προστεθεί από την έκδοση 11 της γλώσσας και μετά. Παρουσιάζονται στατικές και δυναμικές δομές δεδομένων και αλγόριθμοι που λειτουργούν πάνω σε αυτές καθώς και θέματα που έχουν να κάνουν με την εκτίμηση της απόδοσης αλγορίθμων.

\section*{Δομές Δεδομένων}
Οι δομές δεδομένων είναι διευθετήσεις αποθηκευμένων δεδομένων που διευκολύνουν το χειρισμό τους. Τυπικές δομές δεδομένων είναι οι πίνακες, οι λίστες, οι στοίβες, οι ουρές, οι σωροί, οι πίνακες κατακερματισμού, τα γραφήματα και τα δένδρα. Για τις δομές αυτές υπάρχουν διάφορες παραλλαγές τους με ιδιαίτερα χαρακτηριστικά που επιτρέπουν την αποδοτική επίλυση συγκεκριμένων προβλημάτων. Παραδείγματα είναι οι διπλά συνδεδεμένες λίστες, τα ισοζυγισμένα δυαδικά δένδρα, τα κατευθυνόμενα γραφήματα με βάρη καθώς και άλλες δομές.

\section*{Αλγόριθμοι}
Ένας αλγόριθμος είναι μια σειρά λειτουργιών προς εκτέλεση που στοχεύει στην επίλυση ενός προβλήματος. Οι αλγόριθμοι πραγματοποιούν υπολογισμούς, επεξεργασία δεδομένων και εργασίες αυτοματοποιημένης συλλογιστικής. Υπάρχουν αλγόριθμοι που είναι ``διάσημοι'' λόγω του ότι προσφέρουν κομψές λύσεις σε προβλήματα με υψηλή πρακτική και θεωρητική αξία. Μερικοί τέτοιοι αλγόριθμοι είναι ο quicksort για την ταξινόμηση τιμών, ο αλγόριθμος δυαδικής αναζήτησης για τον εντοπισμό μιας τιμής σε μια ταξινομημένη ακολουθία, ο αλγόριθμος εύρεσης των συντομότερων διαδρομών του Dijkstra, οι αλγόριθμοι κρυπτογραφικού κατακερματισμού (cryptographic hashes, π.χ. SHA256) και o αλγόριθμος PageRank των ιδρυτών της Google για την αξιολόγηση της ``σπουδαιότητας'' κάθε ιστοσελίδας σε ένα δίκτυο ιστοσελίδων.  

\section*{Γιατί να μελετήσει κανείς δομές δεδομένων και αλγόριθμους;}
Είναι γεγονός ότι ρυθμός με τον οποίο απαξιώνονται οι γνώσεις στην πληροφορική είναι υψηλός. Τεχνολογίες που κυριαρχούν σήμερα στην έρευνα και στην αγορά μπορεί να μην προσελκύουν ενδιαφέρον μετά από μερικά μόνο έτη. Νέες ιδέες όπως η υπολογιστική νέφους, η εκτύπωση τριών διαστάσεων, η εικονική πραγματικότητα, τα μεγάλα δεδομένα, το διαδίκτυο των πραγμάτων και τα κρυπτονομίσματα οδηγούν σε ανάπτυξη νέων επιστημονικών περιοχών και παραγκωνισμό άλλων. Ωστόσο, οι γνώσεις που αποκτώνται σχετικά με αλγορίθμους και δομές δεδομένων φαίνεται να επιδεικνύουν εξαιρετική αντοχή στο χρόνο. Δεν είναι τυχαίο ότι μερικές από τις σπουδαιότερες ιδέες σχετικά με αλγορίθμους και δομές δεδομένων έχουν διατυπωθεί σε πρώιμες εποχές της πληροφορικής (π.χ. o merge sort το 1945, τα AVL ισοζυγισμένα δένδρα το 1962, οι σωροί το 1964, τα Merkle hash δένδρα το 1979 κ.α.).

Η ποιότητα των λύσεων που παράγονται σε υπολογιστικά προβλήματα εξαρτάται από τους αλγορίθμους και τις δομές δεδομένων που χρησιμοποιούνται. Η γνώση του κατάλληλου αλγορίθμου και των κατάλληλων δομών δεδομένων μπορεί να αποτελέσει τη διαφορά ανάμεσα σε ένα λειτουργικό πρόγραμμα και σε ένα πρόγραμμα που είτε σπαταλά πόρους είτε δεν είναι σε θέση να εφαρμοστεί στην πράξη. Από την άλλη μεριά, η γνώση δομών δεδομένων και αλγορίθμων επιτρέπει τη χρήση κοινής ορολογίας που διευκολύνει την επικοινωνία ανάμεσα στα μέλη της ομάδας ανάπτυξης υπολογιστικών λύσεων. 

Συμπερασματικά, οι δομές δεδομένων και οι αλγόριθμοι είναι ένα εξαιρετικά ενδιαφέρον πεδίο της πληροφορικής με πρακτική σημασία που αντανακλά στην ποικιλία και στην ποιότητα των λύσεων που μπορούν να προταθούν σε υπολογιστικά προβλήματα. Αφορά γνώσεις με υψηλή υπεραξία καθώς δεν αναμένεται, λόγω τεχνολογικών εξελίξεων, να καταστούν παρωχημένες στο μέλλον. Επιπλέον, καθώς πρόκειται για μια επιστημονική περιοχή που αναπτύσσεται επί δεκαετίες, το πεδίο γνώσεων που μπορεί να διερευνηθεί σε θέματα αλγορίθμων και δομών δεδομένων είναι πλούσιο και πολύ καλά τεκμηριωμένο. Συνεπώς, η ουσιαστική ενασχόληση με τις δομές δεδομένων και τους αλγορίθμους αποτελεί αναντικατάστατο εφόδιο για οποιονδήποτε ενδιαφέρεται να αναπτύξει εφαρμογές πληροφορικής. 

\section*{Κώδικας παραδειγμάτων}
Ο κώδικας όλων των παραδειγμάτων βρίσκεται στο \href{https://github.com/chgogos/ceteiep_dsa}{https://github.com/chgogos/ceteiep\_dsa}.
