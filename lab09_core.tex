\section{Εισαγωγή}

\section{Γραφήματα}
Ένα γράφημα ή γράφος (graph) είναι ένα σύνολο από σημεία που ονομάζονται κορυφές (vertices) ή κόμβοι (nodes) για τα οποία ισχύει ότι κάποια από αυτά είναι συνδεδεμένα απευθείας μεταξύ τους με τμήματα γραμμών που ονομάζονται ακμές (edges ή arcs). Συνήθως ένα γράφημα συμβολίζεται ως $G=(V,E)$ όπου $V$ είναι το σύνολο των κορυφών και $E$ είναι το σύνολο των ακμών.

Αν οι ακμές δεν έχουν κατεύθυνση τότε το γράφημα ονομάζεται μη κατευθυνόμενο (undirected) ενώ σε άλλη περίπτωση ονομάζεται κατευθυνόμενο (directed). Ένα πλήρες γράφημα (που όλες οι κορυφές συνδέονται απευθείας με όλες τις άλλες κορυφές) έχει $\frac{|V||V-1|}{2}$ ακμές ($|V|$ είναι το πλήθος των κορυφών του γραφήματος). Αν σε κάθε ακμή αντιστοιχεί μια τιμή τότε το γράφημα λέγεται γράφημα με βάρη. Το γράφημα της Εικόνας 1 είναι ένα μη κατευθυνόμενο
 γράφημα με βάρη.

\subsection{Αναπαράσταση γραφημάτων}
Δύο διαδεδομένοι τρόποι αναπαράστασης γραφημάτων είναι οι πίνακες γειτνίασης (adjacency matrices) και οι λίστες γειτνίασης (adjacency lists).

Στους πίνακες γειτνίασης διατηρείται ένας δισδιάστατος πίνακας $n \times n$ όπου $n$ είναι το πλήθος των κορυφών του γραφήματος. Για κάθε ακμή του γραφήματος που συνενώνει την κορυφή $i$ με την κορυφή $j$ εισάγεται στη θέση $i,j$ του πίνακα το βάρος της ακμής αν το γράφημα είναι με βάρη ενώ αν δεν υπάρχουν βάρη τότε εισάγεται η τιμή 1. Όλα τα υπόλοιπα στοιχεία του πίνακα λαμβάνουν την τιμή 0. Για παράδειγμα η πληροφορία του γραφήματος της Εικόνας 1 διατηρείται όπως φαίνεται στον ακόλουθο πίνακα.

Στις λίστες γειτνίασης διατηρούνται λίστες που περιέχουν για κάθε κόμβο όλη την πληροφορία των συνδέσεών του με τους γειτονικούς του κόμβους. Για παράδειγμα το γράφημα της Εικόνας 1 μπορεί να αναπαρασταθεί με τις ακόλουθες 6 λίστες (μια ανά κορυφή). Κάθε στοιχείο της λίστας για τον κόμβο $v$ είναι ένα ζεύγος τιμών $(w,u)$ και αναπαριστά μια ακμή από τον κόμβο $v$ στον κόμβο $u$ με βάρος $w$.

\section{Παραδείγματα}

\subsection{Παράδειγμα 1}

\subsection{Παράδειγμα 2}


\section{Ασκήσεις}
\begin{enumerate}
\item 
\item 
\end{enumerate}

\begin{thebibliography}{9}
\end{thebibliography}

